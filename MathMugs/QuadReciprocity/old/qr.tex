\documentclass[17pt]{memoir}
\usepackage{amsmath,amssymb}
\pagestyle{empty}
\begin{document}

\begin{center}
  \textbf{Law of Quadratic Reciprocity}
\end{center}

Let $p, q$ be distinct odd primes

Define the \textbf{Legendre Symbol} as

\[
\left( \frac{p}{q} \right) =
  \begin{cases}
    1 & \text{if}\ x^2 \equiv p\ \text{mod}\ q\ \text{for some}\ x \in \mathbb{Z} \\
    -1 & \text{otherwise.}
  \end{cases}
\]

Then

\[
\left( \frac{p}{q} \right) \left( \frac{q}{p} \right)
= (-1)^{\frac{p-1}{2}\frac{q-1}{2}}
\]

\newpage


\begin{center}
  \textbf{Law of Quadratic Reciprocity}
\end{center}

Let $p, q$ be distinct odd primes.

$pRq$ $\rightarrow$ $x^2 \equiv p$ mod $q$, for some $x \in \mathbb{Z}$

$pNq$ $\rightarrow$ $x^2 \not\equiv p$ mod $q$, for any $x \in \mathbb{Z}$

\medskip
\textbf{Case I}

If $p = 1 + 4m$, for some $m \in \mathbb{Z}$,

then either $pRq$ and $qRp$, or $pNq$ and $qNp$.

\medskip
\textbf{Case II}

If $p = 3 + 4m$, $q = 3 + 4n$ for some $m,n \in \mathbb{Z}$,

then either $pRq$ and $qNp$, or $pNq$ and $qRp$.

\bigskip
\begin{flushright}
  \textit{- Carl Friedrich Gauss (1801)}
\end{flushright}


\end{document}
