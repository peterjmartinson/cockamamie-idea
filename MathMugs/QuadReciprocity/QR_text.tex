\documentclass[17pt]{memoir}
\usepackage{amssymb,amsmath,bm}
\usepackage{multirow}
\usepackage[table,xcdraw]{xcolor}
\pagestyle{empty}
\begin{document}

\begin{center}
  \textbf{Law of Quadratic Reciprocity}
\end{center}

Let $p, q$ be distinct odd primes, and define the following symbols:

\medskip
\begin{table}[h]
\centering
  \begin{tabular}{l|l}
  $p\boldsymbol{R}q$ & $x^2 \equiv p\pmod q$ for some $x \in \mathbb{Z}$ \\
   &  $p$ is a \textbf{\textit{quadratic residue}} of $q$ \\
   \hline
  $p\boldsymbol{N}q$ & $x^2 \not\equiv p\pmod q$ for all $x \in \mathbb{Z}$ \\
   &  $p$ is a \textbf{\textit{quadratic nonresidue}} of $q$ \\
  \end{tabular}
\end{table}

\medskip
\textbf{Case I}

If $p = 1 + 4n$, for some $n \in \mathbb{Z}$

then either $pRq$ and $qRp$ or $pNq$ and $qNp$

\medskip
\textbf{Case II}

If $p = 3 + 4m$ and $q = 3 + 4n$ for some $m,n \in \mathbb{Z}$

then either $pRq$ and $qNp$ or $pNq$ and $qRp$

\bigskip
\begin{flushright}
  \textit{--- Karl Friedrich Gauss (1801)}
\end{flushright}

\newpage

\begin{center}
  \textbf{Law of Quadratic Reciprocity}
\end{center}

\arrayrulewidth=2pt
\hspace*{-1cm}
\begin{tabular}{@{}l@{\hskip 8mm}l|cccccccccccccc@{}}
% \begin{tabular}{@{}ll|cccccccccccccc@{}}
                     &                          & \multicolumn{14}{c}{$\boldsymbol{q}$}                                                                                                                                                                                                                                                                                                                                                                                \\[5mm]
                     &                          & 3                         & 5                         & 7                         & 11                        & 13                        & 17                        & 19                        & 23                        & 29                        & 31                        & 37                        & 41                        & 43                        & 47                        \\
\hline
                     & 3                        & --  & N                         & N                         & \textbf{R} & \textbf{R} & N                         & N                         & \textbf{R} & N                         & N                         & \textbf{R} & N                         & N                         & \textbf{R} \\
                     & 5                        & N                         & --  & N                         & \textbf{R} & N                         & N                         & \textbf{R} & N                         & \textbf{R} & \textbf{R} & N                         & \textbf{R} & N                         & N                         \\
                     & 7                        & \textbf{R} & N                         & --  & N                         & N                         & N                         & \textbf{R} & N                         & \textbf{R} & \textbf{R} & \textbf{R} & N                         & N                         & \textbf{R} \\
                     & 11                       & N                         & \textbf{R} & \textbf{R} & --  & N                         & N                         & \textbf{R} & N                         & N                         & N                         & \textbf{R} & N                         & \textbf{R} & N                         \\
                     & 13                       & \textbf{R} & N                         & N                         & N                         & --  & \textbf{R} & N                         & \textbf{R} & \textbf{R} & N                         & N                         & N                         & \textbf{R} & N                         \\
                     & 17                       & N                         & N                         & N                         & N                         & \textbf{R} & --  & \textbf{R} & N                         & N                         & N                         & N                         & N                         & \textbf{R} & \textbf{R} \\
                     & 19                       & \textbf{R} & \textbf{R} & N                         & N                         & N                         & \textbf{R} & --  & N                         & N                         & \textbf{R} & N                         & N                         & N                         & N                         \\
                     & 23                       & N                         & N                         & \textbf{R} & \textbf{R} & \textbf{R} & N                         & \textbf{R} & --  & \textbf{R} & N                         & N                         & \textbf{R} & \textbf{R} & N                         \\
                     & 29                       & N                         & \textbf{R} & \textbf{R} & N                         & \textbf{R} & N                         & N                         & \textbf{R} & --  & N                         & N                         & N                         & N                         & N                         \\
                     & 31                       & \textbf{R} & \textbf{R} & N                         & \textbf{R} & N                         & N                         & N                         & \textbf{R} & N                         & --  & N                         & \textbf{R} & \textbf{R} & N                         \\
                     & 37                       & \textbf{R} & N                         & \textbf{R} & \textbf{R} & N                         & N                         & N                         & N                         & N                         & N                         & --  & \textbf{R} & N                         & \textbf{R} \\
                     & 41                       & N                         & \textbf{R} & N                         & N                         & N                         & N                         & N                         & \textbf{R} & N                         & \textbf{R} & \textbf{R} & --  & \textbf{R} & N                         \\
                     & 43                       & \textbf{R} & N                         & \textbf{R} & N                         & \textbf{R} & \textbf{R} & \textbf{R} & N                         & N                         & N                         & N                         & \textbf{R} & --  & N                         \\
\multirow{-14}{*}{$\boldsymbol{p}$} & 47                       & N                         & N                         & N                         & \textbf{R} & N                         & \textbf{R} & \textbf{R} & \textbf{R} & N                         & \textbf{R} & \textbf{R} & N                         & \textbf{R} & -- 
\end{tabular}

\medskip
\quad\quad\quad\quad\quad \textbf{R} $\rightarrow p$ is a quadratic residue of $q$

\quad\quad\quad\quad\quad N $\rightarrow p$ is a quadratic nonresidue of $q$

\end{document}
